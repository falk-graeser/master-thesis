\documentclass[10pt,a4paper]{book}
\usepackage[utf8]{inputenc}
\usepackage[german]{babel}
\usepackage[T1]{fontenc}
\usepackage{amsmath}
\usepackage{amsfonts}
\usepackage{amssymb}
\usepackage[left=2cm,right=2cm,top=2cm,bottom=2cm]{geometry}
\author{Falk Gräser}
\title{Entwicklung von Visualisierungs- und Integrationsansätzen für Continuous-Integration und Feature-Branches}
\begin{document}

\maketitle

\newpage

\chapter{Einleitung}

\chapter{Problemanalyse}

In diesem Kapitel muss ich darstellen was Contiuous Integration bedeutet, wofür es verwendet wird und welche Tücken es mit sich bringt.
Ebenso für Feature Branches.
Und ich muss darstellen, warum Martin Fowler darüber geschrieben hat, dass 

"When you isolate the feature branches, there is a risk of a nasty conflict growing without you realizing it"
https://martinfowler.com/bliki/FeatureBranch.html


\section{Continuous Integration}

\begin{itemize}
\item Warum Continuous Integration
\item Stärken von CI
\item Probleme mit Continuous Integration
\end{itemize}

\section{Feature Branches}

\begin{itemize}
\item Warum Feature Branches
\item Stärken von FB
\item Schwächen von FB
\end{itemize}

\chapter{Visualisierung und Methodiken}

\begin{itemize}
\item Metriken um Branches zu beurteilen
\item automatisierte Releasekombination
\item automatisierte Konflikterkennung
\item Regeln, wie nur FastForward-Merges erlauben
\end{itemize}

\chapter{Zusammenfassung}


\end{document}