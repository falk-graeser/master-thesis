\documentclass{article}
\usepackage{graphicx}

\begin{document}

\title{Dezentrales Continuous Deployment: Werden Akzeptanz, Transparenz und Qualit\"at der Softwareentwicklung durch den Einsatz verbessert?}
\author{Falk Gr\"aser}

\maketitle

\begin{abstract}
Continuous Integration ist, sp\"atestens seit der Ver\"offentlichung \glqq{}Extreme Programming Explained\grqq{} von Kent Beck im Jahre 1999, eine anerkannte Basis f\"ur hochqualitative Softwareentwicklung. Knapp 20 Jahre nach dem Werk von Kent Beck ist Continuous Integration eine g\"angige Praxis und wird als Basis f\"ur Continuous Deployment genutzt. Die damit einhergehenden Forderungen an die F\"ahigkeiten der Entwickler und das notwendige Vertrauen in den automatisierten Prozess erzeugen h\"aufig Widerwillen und Skepsis.

W\"ahrend die notwendige Infrastruktur w\"achst und sich an die Arbeitsweise und das Projektfeld des Teams anpasst, werden immer mehr strukturelle Abh\"angigkeiten geschaffen. Diese k\"onnen im Laufe des Lebenszyklus von Anwendung, Infrastruktur und Unternehmen f\"ur zahlreiche Herausforderungen zwischen den Teams, aber auch zwischen Administration und Team sorgen.
W\"ahrend es in der Softwareentwicklung von zentraler Wichtigkeit ist zu entkoppeln und zu modularisieren, bauen sich gerade in der Infrastruktur h\"aufig monolithische, zentrale Konstrukte auf.

Ziel dieser Arbeit ist es daher traditionelle, zentrale Continuous Integration und Continuous Deployment Ans\"atze zu analysieren und deren St\"arken und Schw\"achen mit denen verteilter Ans\"atze zu vergleichen.

Besonderes Augenmerk soll dabei auf die Eigenverantwortung des Entwicklers und die prozessgesicherte Qualit\"at des Softwareproduktes durch Tests und Metriken gelegt werden.
Im Rahmen einer produktiv genutzten, individuellen Softwarel\"osung soll zudem der bestehende Deployment-Prozess analysiert, dezentral konzipiert und prototypisch umgesetzt werden.

Die Masterarbeit soll sich zudem mit folgenden Thesen auseinander setzen:
\glqq{}Continuous Deployment ist zu komplex um es in jedem Projekt zu verwenden.\grqq{},
\glqq{}Der Einsatz von Continuous Deployment steigert die Qualit\"at des Entwicklungsprozesses.\grqq{}, und
\glqq{}Es kann eine einheitliche Continuous-Deployment-L\"osung geben, welche mithilfe einer einheitlichen, internen DSL in jedes Projekt integrierbar ist.\grqq{}
Dabei sollen Indizien oder sogar Beweise jeweils f\"ur deren Best\"atigung bzw. Falsifizierung geliefert werden.
\end{abstract}

\end{document}