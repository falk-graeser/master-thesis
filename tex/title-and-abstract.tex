\documentclass[11pt,english,german]{article}
\usepackage{graphicx}
\usepackage[T1]{fontenc}
\usepackage[utf8]{inputenc}
\usepackage[ngerman]{babel}
\usepackage[mmddyyyy]{datetime}
\usepackage[inner=4cm,outer=2cm,bottom=3.5cm]{geometry}
\geometry{a4paper}

\setlength{\oddsidemargin}{15.5pt}
\setlength{\evensidemargin}{15.5pt}

\setlength{\parindent}{4mm}
\setlength{\parskip}{1mm }

\begin{document}

\sffamily

\title{\vspace{-2.0cm}\textbf{Entwicklung von Visualisierungs- und Integrationsansätzen für Continuous-Integration und Feature-Branches}}
\author{Falk Gräser}
\date{01.12.2017}

\maketitle
\pagenumbering{gobble}

\paragraph{\textsf{Aufgabenstellung}}
Continuous Integration ist, spätestens seit der Veröffentlichung \glqq Extreme Programming Explained\grqq{}  von Kent Beck im Jahre 1999,
eine anerkannte Basis für hochqualitative Softwareentwicklung. Knapp 20 Jahre nach dem Werk von Kent Beck ist Continuous Integration weit
verbreitet und wird vollautomatisiert für Continuous Deployment genutzt.
\\
\\
Kern der Continous Integration ist das häufige Zusammenführen aller aktiven Arbeiten. Angelehnt an das \glqq Subversion\grqq{}-Versionskontrollsystem
wird daher von auch \glqq Trunk Based Development\grqq{} gesprochen.
Allerdings fordert gerade diese andauernde und nicht zu umgehende Integration, ein hohes Maß an Kommunikation und versierten Softwareentwicklern.
Mit dem Erstarken von dezentralen Versionierungssystemen (DVCS) gab es eine alternative zur stetigen Integration. Für jede Anforderung (Feature)
konnte nun unter vergleichsweise geringem Aufwand ein eigener Arbeitszweig (Branch) erstellt werden. In der Theorie konnten
dann alle Arbeitszweige \glqq einfach\grqq{} integriert und veröffentlicht werden.
\\
\\
Während viele kommerzielle Lösungen dieses Verfahren anpreisen und zahlreiche Werkzeuge das Vorgehen unterstützen, gibt
es doch immer noch deutliche Schwierigkeiten mit der Umsetzung. Häufig entwickeln sich \glqq kurzlebige und kleine\grqq{}
Entwicklungszweige zu \glqq großen und langlebigen\grqq{} Veränderungen. Diese betreffen schnell verschiedene Komponenten
der Anwendung und werden immer schwieriger zu integrieren.
\\
\\
Die daraus entstehenden Diskussionen führten 2009 dazu, dass der bekannte Softwareentwickler Martin Fowler sich recht
deutlich gegen \glqq Feature Branches\grqq{} aussprach.
\\
\\
Ziel der Arbeit ist es nun den Konflikt von \glqq Feature Branches\grqq{}  und \glqq Continuous Integration\grqq{}
aufzuarbeiten, sowie visuelle und methodische Lösungsansätze anzubieten.
Weiteres Augenmerk wird dazu auf die Verbindung zu \glqq Continuous Deployment\grqq{}  und die damit einhergehenden
notwendigen Metriken und Kriterien, für eine automatisierte und erfolgreiche Bereitstellung von Software gelegt.
\\
\\
Die Masterarbeit soll sich zudem mit folgenden Thesen auseinander setzen: \\
\glqq Feature Branches und Continous Integration sind unvereinbare Prinzipien.\grqq{},\\
\glqq Continuous Deployment ist zu komplex um es in jedem Projekt zu verwenden.\grqq{} und\\
\glqq Der Einsatz von Continuous Deployment steigert die Qualität des Entwicklungsprozesses.\grqq{}\\
Dabei sollen Indizien oder sogar Beweise jeweils für deren Bestätigung bzw. Falsifizierung geliefert werden.

\end{document}