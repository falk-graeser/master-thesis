\chapter{Zusammenfassung}

während ci die last der integration auf den Schultern aller aushandelt, verlagert fb die integration auf den schultern des \glqq konfliktverursachers\grqq{} somit wird eine kommunikation gefördert, allerdings ohne blockierende nachteile (rote ci ampeln als anti-pattern)

unabhängig von ci und fb sollten systeme clean code ansätze nutzen um gut skalierbare, wartungsarme systeme zu erstellen


mergen, validieren, bewerten

prüft entwickler in ci nicht seinen commit, behindert er alle die aktualisieren

- konflikt kann nicht gelöst werden
- konflikt kann durch technische unterstützung gemildert werden
- konflikt kann durch strikte regelungen gemindert werden
- merges können teilweise nicht automatisch gelöst werden, review muss immer manuell durchgeführt werden
- nur die softwarevalidierung kann letztendlich sicherstellen, dass die software die anforderungen erfüllt und keine fehler gefunden werden konnten
- viele konflikte können auf allgemeine problem der softwareentwicklung zurück geführt werden


ausblick
- standardisiertes cd
- semantisches mergen
- ide tools mit merge aufforderungen