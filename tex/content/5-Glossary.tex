\makeglossaries

\newglossaryentry{Code-Kopplung}{name={Code-Kopplung}, description={
	Code-Kopplung beschreibt die Abhängigkeiten von Komponenten. Diese ist zwischen zwei Komponenten umso stärker, je schwerer die Implementierung durch eine andere getauscht werden kann, ohne die zweite Komponente anzupassen
}}
\newglossaryentry{Commitment}{name={Commitment}, description={
	ist eine gemeinsame Vereinbarung. Der Begriff wurde durch agile Methodiken geprägt. Diese erfordern zu Beginn eines Inkrements, das Commitment aller Stakeholder
}}
\newglossaryentry{Merge}{name={Merge}, description={
	ist die Kombination zweier Commits in einer Versionsverwaltung
}}
\newglossaryentry{Metrik}{name={Metrik}, description={
	ist eine Funktion oder ein Verfahren, um einen Messwert zu ermitteln
}}
\newglossaryentry{Refactoring}{name={Refactoring}, description={
	beschreibt das gezielte Verändern eine Code-Abschnitts in einer Softwareanwendung. Ziel ist in der Regel die Verbesserung der Code-Qualität
}}
\newglossaryentry{Release}{name={Release}, description={
	ist die Bezeichnung für die Veröffentlichung einer Reihe an Änderungen.
}}
\newglossaryentry{Repository}{name={Repository}, description={
	beschreibt in Verbindung mit Versionsverwaltungssystemen eine Ablage für Commits
}}
\newglossaryentry{Review}{name={Review}, description={
	,genauer Code-Review, ist ein Verfahren, um entwickelten Software-Code zu begutachten und zu bewerten. Ziel ist die Verbreitung von Wissen über die Softwareanwendung und die Verbesserung der Code-Qualität
}}
\newglossaryentry{Test-Suite}{name={Test-Suite}, description={
	bezeichnet eine Sammlung von Testfällen. Häufig existieren mehrere Test-Suits, die nach Test-Kategorien getrennt sind.
}}
\newglossaryentry{Workflow}{name=Workflow,description={
	ist ein Arbeitsauflauf, der eine definierte Abfolge von Arbeitsschritten enthält.
}}
\newglossaryentry{Peer-To-Peer}{name={Peer-To-Peer}, description={
	beschreibt eine Netzwerkverbindung. Die Netzwerkverbindung für direkt zwischen zwei Teilnehmern ausgehandelt ohne den Zugriff über Dritte zuzulassen.
}}
\newglossaryentry{Coding-Guidelines}{name={Coding-Guidelines}, description={
	sind eine Reihe an Richtlinien, welche für die Softwareentwicklung aufgestellt werden können. In der Regel legt sich ein Softwareentwicklungsteam auf gemeinsame Coding-Guidelines fest, um das gegenseitige Verständnis des Software-Codes zu verbessern.
}}