\chapter{Problemanalyse}

Die Zusammenführung der Arbeit von mehr als einem Entwickler ist ein komplexer und nicht trivialer Vorgang. \\
Viele Teile eines Softwareprojektes beeinflussen andere und häufig werden an Schnittstellen in Programmen Änderungen von mehreren Entwicklern vorgenommen. \\
Es muss sichergestellt werden, dass die Änderungen zusammengeführt werden können und dass das Ergebnis syntaktisch und semantisch korrekt ist, sowie den Anforderungen entspricht.

Während die syntaktische und semantische Validierung klassische Themen der theoretischen Informatik sind, erfordert die Erhebung und Aufbereitung von Anforderungen eine ganz eigene Betrachtung durch ein konkretes Anforderungsmanagement.

Gerade wenn die Software einen gewissen Umfang übersteigt, ist daher ein solides Anforderungsmanagement essenziell. Zudem müssen Mechanismen geschaffen werden, um sicherzustellen, dass diese Anforderungen über die komplette Laufzeit der Software erfüllt bleiben. 

Leider sieht die Realität in der Softwareentwicklung nicht selten anders aus. Fehlende, nicht dokumentierte oder veraltet Anforderungen sind keine Seltenheit.\\
Ebenso sind lange Zeitintervalle in denen sich Teile der Software in keinem lauffähigen oder einem fehlerbehaftetem Zustand befinden häufig anzutreffen.

Fehlende Anforderungen und fehlerhafte Umsetzungen werden häufig erst zum Zeitpunkt des Testes - oder schlimmer - zum Zeitpunkt der Auslieferung offenbart.

Die Gründe für diese Schwierigkeiten sind häufig in kein homogenes Bild zu bringen. Vielmehr tragen verschiedene Faktoren dazu bei, die erst in der Kombination zu erheblichen Problemen führen können.

\section{Vorgehensmodelle}

Um komplexe Software in einer strukturierten und definierten Herangehensweise zu erstellen, wurden ausführlich beschriebene und wohl definierte Vorgehensmodelle definiert. Ziel war eine klare Schrittfolge mit eindeutigen Zielstellung, wie man es aus anderen Ingenieursdisziplinen gewohnt war. 

Die daraus entstehenden Modell hielten sich meist an eine klare Folge aus Planungsphase,  Konzeptions-, Implementierungs-, Test- und Abnahmephase. Beispiele sind hier das Wasserfallmodell und das V-Modell. Gerade in bereichen höchster Sicherheit und einer starken Bindung an bürokratische Strukturen wird auch heute noch das V-Modell erfolgreich eingesetzt[quote Referenz Bund?].

Mit der komplexität einer Software steigt in der Regel auch der Umsetzungsaufwand und die damit verbunden Zeit. Durch die klare Phasenteilung in den klassischen Vorgehensmodellen, kann daher von der Anforderungsaufnahme bis zu Auslieferung einiges an Zeit verstreichen. Nicht selten ändern sich allerdings Anwendungsszenarien und Benutzeranforderungen an das entstehende System. Eine weitere Schwierigkeit ist die Prüfung der Umsetzung von Anforderungen. Sollte die Umsetzung einer Anforderung, zum Beispiel durch ungenaue oder mehrdeutige Angaben, nicht den Maßgaben des Tests entsprechen, vergehen Wochen oder Monate bis diese Diskrepanz aufgedeckt wird.

Immer wieder kam es daher zu unbefriedigenden oder sogar gescheiterten Softwareprojekten. Gerade bei komplexen Softwaresystem mit einem entsprechenden Umsetzungsaufwand, ist der Schaden dann sehr hoch.

Um diesen Schaden zu minimieren entwickelte sich eine alternative Herangehensweise, welche vor allem Interaktion, Funktionalität, Kundenorientierung und Veränderungsbereitschaft als wichtig erachtet. Festgehalten im Agilen Manifest[quote manifest] proklamierten viele bekannte und angesehene Softwareexperten ihren Willen Softwareprojekte grundsätzlich anders zu fokussieren.

Beeinflusst von der Lean-Bewegung aus der Automobil-Fertigungsindustrie\footcite{kent1999} wollte man die Produktivität deutlich steigern und aus den bürokratisch anmutenden Vorgehensmodellen ausbrechen.

Das Agile Manifest stellt hohe Ansprüche an Entwickler und erzwingt ein Umdenken im Umgang mit dem Projektprozess. Während in traditionellen Vorgehensmodellen jede Phase eine längere zeitliche Periode einnimmt, so werden diese Phasen und Zeiträume in agilen Vorgehen deutlich verkürzt, um damit einhergehende Informationsflüsse deutlich zu beschleunigen.
Diese kurzen Zeiträume und damit einhergehenden vielen Iterationen geben kontinuierlich und schnell Auskunft über den Zustand eines Projektes. Außerdem wird es einfacher zu erkennen, welche Schritte tatsächlich wertvoll sind und welche Schritte dem Projektfortschritt eher abträglich sind. 

Oberstes Ziel hierbei ist die Lieferung von ``wertvoller'' Software, also die Einschätzung der Anforderungen mit den Stakeholdern und priorisierte Lieferung dieser Anforderungen.

Dieses Vorgehen führt automatisch dazu, dass frühzeitig deutlich wird, ob die richtigen Anforderungen als wichtig erkannt wurden und welche Anforderungen implizit angenommen und daher im verborgenen lagen.

Ein weiterer Aspekt ``wertvoller'' Software ist ihre Funktionalität. Diese sicherzustellen in einer kontinuierlichen Auslieferung erfordert ein hohes Maß an Qualitätssicherung. Schon sehr früh im Entwicklungsprozess eines iterativ wachsenden Systems wird deutlich, dass keine menschliche Resource mehr im Stande ist, alle Anforderungen mit jeder Auslieferung zu testen und die Qualität sicherzustellen. Die Automatisierung dieser Tests ist somit unausweichlich.

\section{Qualitätssicherung und Softwaretest}

Die Qualitätssicherung einer Software muss als oberste Prämisse sicherstellen, dass alle Anforderungen erfüllt sind und dass die Anforderungen auch erfüllt bleiben. Nicht selten führen Neuerungen in einer Software dazu, dass bereits bestehenden Funktionalität beeinträchtigt wird. 
Aufbauend auf der Annahme, dass alle notwendigen Anforderungen beschrieben sind, müssen diese in unterschiedlichen Ebenen überprüft und sichergestellt werden.
Diese Ebenen richten sich danach welche Anforderungen und damit verbundenen Merkmale geprüft werden müssen, aber auch danach wie komplex die damit verbundene Prüfung ist.

In der Literatur werden meist Unit-, Funktions- und Integrationstests, sowie System- und Abnahmetests (Akzeptanztest) unterschieden.
Die Unterscheidung liegt hier in der Größe des zu betrachtenden Ausschnitts der Software und damit der zu prüfenden Aussage. Während in einem Abnahmetest eine direkt Verbindung zwischen Anforderung und Testfall hergestellt werden kann, so ist dies in einem einzelnen Unittest meist nicht mehr möglich.
\paragraph{Akzeptanztest}
Akzeptanz- oder auch Abnahmetests stellen in erster Linie die Erfüllung der definierten Anforderungen sicher. Akzeptanztests werden häufig in der Form von Anwendungsszenarien (Use-Cases) beschrieben und sollte sicher daher nur auf das Zusammenspiel von Systemen und Akteuren beziehen. Die Verwendung von technische Details der Software, wie Datenbankspezifikationen oder konkrete Systemimplementierungen sollten vermieden werden. Nicht nur erhöht es den Wartungsaufwand dieser Tests, es verschiebt auch den Fokus des Tests weg von den eigentlich zu prüfenden Merkmalen und Aktionen.

Nicht notwendiger Weise, aber häufig dauert die Durchführung der Akzeptanztests recht lange. Daher werden sie meist seltener ausgeführt. Die lange Ausführungszeit macht sie zudem unhandlich, um eine schnelle Validierung eines in Entwicklung befindlichen Softwarestückes zu gewährleisten.

\paragraph{Funktions- und Integrationstests}

Funktions- und Integrationstests dienen der Überprüfung von Schnittstellen. Es wird sichergestellt, dass verschiedenen Teile der Software, verschiedene Komponenten, nach den vereinbarten Schnittstellen miteinander kommunizieren.
Neben der Absicherung von Seiteneffekten zu anderen Komponenten und Fremdmodulen, bieten Integrationstests auch eine gute Beschreibung einer Komponente und derer Schnittstellen.

\paragraph{Unittests}

Unittests bilden die unterste Ebene der Softwaretests. Maßgeblich bei Unittests ist die völlige Abkopplung von anderen Teilen, als des zu testenden Teils (Unit) der Software. Jegliche Interaktion mit externen Services sollte strikt vermieden werden.
Ein ordnungsgemäß verfasster Unittest stellt neben der Regressionssicherheit eines Moduls auch einen Teil dessen Dokumentation sicher. Da ein Unittest das zu testende Modul initialisiert und die notwendigen Schnittstellen bereitstellt, bildet er auch anschaulich die Intention des Moduls ab.

\section{Automatisierung der Softwareerstellung}

Die Bereitstellung einer Software baut auf mehreren Faktoren auf. Es wird ein konkreter Stand der Softwarequellen benötigt, ein ausführendes System und es muss für eine Übersetzung der Softwarequellen zu dem System gesorgt werden.
Je nach Anforderung der Software und des Ausführungsszenarios reicht dieser Vorgang von einer trivialen Kopie zu einem mehrstufigen, komplexen Vorgang. 

Im nachfolgenden wird beschrieben wie Software in einen wohldefinierten, funktionalen Zustand gebracht und dieser validiert wird. Dazu wird auf Versionsverwaltungssysteme, Konfigurationsmanagement, die automatisierte Erstellung und deren Test eingegangen.

\subsection{Kodeverwaltung und Versionsverwaltungssysteme}

Bereits frühzeitig wurde in der Bereitstellung von Software in Programmversionen unterschieden. Anwender konnte so transparent nachvollziehen, wie fortschrittlich die Anwendung war, die sie erwerben wollten.
Für die Entwicklung ergab sich daraus vor allem die Herausforderung nachfolgende Fehlerkorrekturen für viele verschiedene Entwicklungszustände bereit zustellen.
Eine nachvollziehbare und wartungsarme Ablage der einzelnen Versionsstände war daher unabdingbar.

Auch wenn es möglich ist alle Programmversionen in manuell gepflegten Ablagen zu verwalten, gelangt die Entwicklung einer komplexen Software schnell an den Punkt an dem eine Ablage zusätzliche Bedingungen erfüllen muss.

Das dabei herausstechende Merkmal ist die Zustandssicherheit. Es muss gewährleistet sein, dass der Quellcode für eine bestimmte Version in einem konkreten, definierten Zustand ist. Als weiteres Merkmal ist die Transparenz anzuführen. Transparenz ist vor allem dann von Nöten, wenn mehr als ein Entwickler an einer Quellcodesammlung arbeitet. Schnell geht die Übersicht verloren, welche Änderungen wann, von wem und warum eingebracht wurden. Oftmals sind diese Informationen aber hilfreich um Entscheidungen im Entwicklungsprozess zu treffen.

Über die Entstehung von Versionsverwaltungssysteme[quote Soft.Quali] haben sich vor allem zwei Prinzipien herausgebildet: Zentrale und Dezentrale Versionsverwaltungssysteme. Zwar unterscheiden sich die Versionsverwaltungssysteme auch in anderen Merkmalen, zum Beispiel in der Ablage der Quellen und deren Änderungen, diese Merkmale haben aber nahezu keinen Einfluss auf die Verwendung im Softwareentwicklungsprozess.

Auch wenn heutzutage die dezentralen Versionsverwaltungssystem stark zunehmen, so fällt bei näherer Betrachtung auf, dass diese oft primär als zentrale Versionsverwaltungssysteme verwendet werden.

\paragraph{Zentrale Versionsverwaltungssysteme} folgen dem Prinzip, dass es nur einen zentralen Punkt gibt auf dem alle Versionstände gespeichert werden. Dieses ``single point of truth'' Prinzip sorgt für einen einfachen Kontrollfluss, welcher dafür sorgt, dass alle Partizipierer mit einfachen Mitteln auf den gleichen Versionsständen arbeiten. Damit einhergehend ist eine Übersicht aller Änderungen problemlos möglich.

Nachteil dieser Variante ist häufig der Initialaufwand, da eine zentrale Komponenten bereitgestellt werden muss. Zudem sind zentrale Systeme generell störanfälliger und Probleme betreffen immer alle Teilnehmer gleichzeitig.
Je nach verwendeter Versionierungsstrategie kann es zudem zu einer zentralen Blockade ganzer Bereiche der Softwarequellen kommen, wenn diese für eine exklusive Bearbeitung gesperrt sind.

\paragraph{Dezentrale Versionsverwaltungssysteme} benötigen im Gegensatz keine zentrale Instanz und können problemlos in einem losen Verbund von Einzelsystemen betrieben werden. 
Im Gegensatz zur zentralen Verwendung, existierend dadurch deutlich mehr Zustände, die ohne entsprechende Maßnahmen nicht zu überblicken sind. Dieser Mehraufwand bringt allerdings eine Flexibilität mit sich, welche viele Möglichkeiten bietet um Softwareänderungen zu koordinieren. Änderungen können in ihrer Reihenfolge transparent manipuliert und der Zeitpunkt der Zusammenführung beliebig gewählt werden. 
Im Bereich ``Feature Branches'' werden Versionierungsstrategien im Detail erläutert.

Ihrer dezentralen Natur geschuldet, sind sie ausfallsicherer als die zentralen Versionsverwaltungssysteme, benötigen allerdings die entsprechenden Informationen und Befugnisse zur Verbindung, um diesen Vorteil nutzen zu können.

\subsection{Konfigurationsverwaltung}
\label{subsec:konfigurationsverwaltung}

Konfigurationsverwaltung ist eine wichtige Basis für die automatisierte Erstellung eines Softwareprojektes. Sie definiert wie alle Artefakte in einem Softwareprojekt orchestriert werden. Im Detail wird dabei Erstellung, Identifikation, Validierung und Ablage geregelt. Neben den Projekteigenen Artefakten müssen aber auch deren Umgebung genau definiert sein. \footcite{humble2010}

Ein lockerer Umgang mit der Konfigurationsverwaltung kann gerade zu Anfangs ohne nennenswerte Folgen einhergehen. Spätestens aber während der Reifung und Alterung eines Softwareprojektes sind Folgen deutlich zu spüren. Gerade wenn sich unbemerkt die Version oder Konfiguration der Software eines Drittanbieters ändert, kann dies zu sehr schwer zu identifizierbaren Problemen führen. In anderen Fällen reicht es allerdings auch, wenn sich die Konfiguration zwischen Entwicklungs- und Produktivumgebung leicht unterscheidet. Auch hier können damit zusammenhängende Fehler unnötig viel Zeit in Anspruch nehmen.

Im Idealfall kann eine vollständige Konfigurationsverwaltung:
\begin{itemize}
\item eine Umgebung bereitstellen, die genau definiert welches Betriebssystem, in welcher Version, installiert ist und welche Anwendungen, mit welcher Konfiguration, darauf liegen
\item auf jeder Umgebung jeden dieser Bestandteile einzeln inkrementell verändern und diese Änderungen auch synchronisiert bereit stellen.
\item alle Änderungen mit Zeitpunkt und Verursacher einfach darstellen
\item Sicherheitsregeln und Konventionen über alle Systeme sicherstellen
\item alle diese Funktionalitäten barrierefrei dem Entwicklungsteam zur Verfügung stellen
\end{itemize}

Alle Forderungen sind sicherlich nicht immer Notwendig, sorgen aber dafür, dass Projekte langfristig, mit geringem Aufwand betreut werden können. Zudem wird deutlich, dass Virtualisierung und Versionsverwaltung[quote cont.deliv] der entsprechenden Komponenten den einhergehenden Aufwand deutlich verringern können.

\subsection{Abhängigkeitsverwaltung}

In der Abhängigkeitsverwaltung werden die Softwareabhängigkeiten der Anwendung beschrieben. In auf  Komponenten basierenden Anwendungen schließt dies die eigenen, interne Abhängigkeiten mit ein. Diese Abhängigkeiten, meist Module genannt, sollen die mit monolithischen Strukturen einhergehenden Nachteile unterbinden. Monolithen leiden häufig unter einer schwachen Softwarearchitektur und entwickeln sich daher entsprechend schlecht weiter. Mögliche Skalierungen des Systems oder ein Austausch von einzelnen Bestandteilen sind mit einem erheblichen Mehraufwand verbunden.

Ist eine Anwendung hingehen auf mehrere Komponenten aufgeteilt, kommen neben den genannten Vorteilen der Skalierbarkeit und Austauschbarkeit, auch Vorteile für das verteilte Arbeiten von Teams. Allerdings führt eine Komponenten basierende Anwendung natürlich auch zu einem organisatorischem Mehraufwand. Dieser Mehraufwand besteht in der Beschreibung der Abhängigkeiten und der Auflösung des damit entstehenden Abhängigkeitsgeflechtes. Übliche Abhängigkeitsverwaltungen verwenden hierzu Versionsnummer aus der Versionsverwaltung, sowie eine Paketbeschreibung in einem eigenen Format. 

Für die automatisierte Erstellung der Softwareanwendung ist es erforderlich, ähnlich wie bereits bei der Versionsverwaltung, ein eindeutiges Abhängigkeitsbild zu beschreiben. Während zum Auflösen der Abhängigkeiten meist mehrdeutige Ausdrücke verwendet werden, die eine Bandbreite an Versionen der Abhängigkeiten erlauben, ist für den Erstellungsvorgang wichtig die konkret verwendeten Versionen zu dokumentieren und mit dem erstellten Produkt auszuliefern. Ähnlich der erwähnten Paketbeschreibung, wird dies mit einer Lock-Datei gesichert.

\subsection{Automatisierte Erstellung und Auslieferung}

Der an sich triviale Schritt, der Erstellung (Build) einer Anwendung, birgt gewisse Tücken. So kann die Ausführung eines komplexen Erstellungsskriptes einen sehr langen Zeitraum in Anspruch nehmen und ein besonders kompliziertes Skript kann die Arbeit damit intransparent gestalten und das Entwicklungsverhalten deutlich benachteiligen. [quote paper] Werden bei einem Erstellungsskript Aspekte der vorangegangen Abschnitte zu Versions-, Konfigurations- und Abhängigkeitsverwaltung nicht berücksichtigt, können zudem inkonsistente und damit nicht wiederholbare Resultate erzeugt werden. Diese können leicht zu schwer oder nicht auffindbaren Fehlern führen.

Ziel bei der Erstellung eines Erstellungsskriptes ist es folglich einen konsistenten und transparenten Ablauf zu befolgen. Dies wird unter anderem durch die korrekte Versionsverwaltung der einzelnen Komponenten, Konfigurationen und Abhängigkeiten erreicht. Mit diesen Informationen können erstellte Produkte, auch ``build artifacts'' oder Artefakte genannt, in einem zentralen Artefakt-Lager abgelegt werden. Dies gewährleistet sowohl einen schnellen Zugriff auf Teilartefakte, als auch einen konsistentes Ergebnis bei der Erstellung der Auslieferung.

Die Auslieferung der Software ist direkt mit der Erstellung verwoben. Bei jeder Auslieferung muss sichergestellt werden, dass Software auf dem entsprechenden Zielsystem auch lauffähig ist und dass sie richtig konfiguriert ist für das Zielsystem. Ein vollständiges Konfigurationsmanagement beschreibt die Systemabhängigkeiten und sorgt folglich für die korrekte Auslieferung der richtigen Artefakte.

Sowohl bei der Erstellung, als auch bei der Auslieferung ist insbesondere auf eine vollständige Beschreibung und die Einhaltung dieser zu achten. Gerade auf Entwicklungssystemen können Abhängigkeiten vorhanden sein, welche zwar für die Funktionalität der Anwendung sorgen, allerdings nicht beschrieben wurden. Bei der Auslieferung auf andere Systeme können diese ``Schattenabhängigkeiten'' dann zu erwartetem Verhalten oder Abstürzen führen. Ein sehr kontrollierter Ausweg aus dem Dilemma von mehrfach genutzten Entwicklungsumgebungen bietet die Virtualisierung der verwendeten Systeme. Wie bereits für Softwareabhängigkeiten der Anwendung beschrieben, können diese Virtualisierungen als Artefakte bereitgestellt werden und unterstützen Konsistenz, Transparenz sowie Skalierbarkeit[quote virtualiserung mit docker].

Bei der Auslieferung der Software können grob zwei Stufen der Automatisierung unterschieden werden. Eine Stufe ist die manuell angestoßene Auslieferung, bei der der Zeitpunkt der Auslieferung davon bestimmt wird, wann das automatisierte oder automatisiert-moderierte Script ausgeführt wird. Eine Andere ist die vollautomatisierte Auslieferung, bei der der Zeitpunkt lediglich von der Durchlaufzeit einer Änderung durch die Qualitätskontrolle bestimmt wird.
Des weiteren werden unterschiedliche Auslieferungsstrategien (``deployment strategies'') unterschieden, es seien an dieser Stelle nur exemplarisch das ``blue green deployment'' und das ``canary deployment'' genannt. Die Auslieferungsstrategien werden dabei je nach Anforderungen ausgewählt und haben Vorzüge wie Ausfallsicherheit, Bewertung der ausgelieferten Änderung oder Skalierbarkeit der Zielsysteme.

Da die konkrete Auslieferung auf die Bereiche ``Continuous Integration'' und ``Feature Branches'' nur begrenzt Einfluss hat, verweise ich an dieser Stelle auf die entsprechende Fachliteratur.

\subsection{Validierung und Test}

Die Validierung einer Software ist ein unglaublich komplexes Unterfangen. Die vollständige Richtigkeit eines Programmes zu beweisen, scheitert bereits bei kleinen Anwendungen, da die Programmkomplexität schlicht zu schnell steigt. Zudem können nur Teile validiert und getestet werden, welche im Vorfeld bestimmt worden sind. Unklare, mehrdeutige oder fehlende Anforderung machen einen Test unvollständig oder nehmen im die Aussagekraft[quote software quali p243 kap4.6].

Da eine Software nicht vollständig getestet werden kann, muss eine Strategie gewählt werden, welche Tests verwendet werden und mit welchem Ziel. Zudem können Metriken gewählt werden, die eine Aussage über die Testvollständigkeit liefern können. Bei der Entscheidung welche Strategie gewählt wird, fallen verschiedene Kriterien ins Gewicht. Zum einen muss betrachtet werden, welche Anforderungen mit welchem Testverfahren getestet werden können, zum anderen welche Granularität für jeden Test verwendet wird.

Insbesondere in der automatisierten Erstellung sind diese Entscheidungen wichtig, da sie eine Basis liefern für die Aussage, wie schnell die Testsammlung ausgeführt werden kann. Und damit einhergehend, wie schnell kann eine Aussage getroffen werden, ob ein Inkrement der Software valide ist. 
Ein wichtiger Punkt ist hier auch die aktive Wartung der Testsammlung, jeder Fehler der nicht von der Testsammlung erfasst wurde, muss spätestens nach der Beseitigung des Fehlers auch durch die Testsammlung validiert werden. Diese Regressions-Sicherheit stellt in vielen Testsammlungen häufig den größten Teil der Tests [quote software quali].

Da Tests und Metriken die einzigen Möglichkeiten sind um automatisch Einschätzungen der Software vorzunehmen, werden diese Bereiche erneut im Kapitel~\ref{ch:visu_meth} aufgegriffen und näher auf deren Wertschöpfung eingegangen.

\section{Continuous Integration}

Continuous Integration wurde sehr treffend beschrieben \footcite{fowler2006}
\blockquote {Continuous Integration is a software development practice where members of a team integrate their work frequently, usually each person integrates at least daily - leading to multiple integrations per day. Each integration is verified by an automated build (including test) to detect integration errors as quickly as possible.}

Continuous Integration wurde erstmal von Kent Beck als Teil einer Reihe von Praktiken des Entwicklungsvorgehens ``Extreme Programming''\footcite{kent1999} beschrieben. Die Idee die Integration von Codeständen zu fördern, in dem man die Entwickler anhält ihre Änderungen täglich oder häufiger abzugleichen und durch einen automatisierten Vorgang als lauffähig zu beweisen. Damit konnten Fehler zum Zeitpunkt des Auftretens gefunden werden, insbesondere Fehler, welche erst durch die Integration mit den Änderungen anderer Entwickler auftreten. Damit fordert Continuous Integration zugleich auch eine aussagekräftige Testabdeckung mit automatischen Tests und dass Entwickler einen entdeckten Fehler direkt beheben, da sonst alle anderen Teammitglieder behindert werden.

Natürlich nutzt Continuous Integration auch alle Vorteile, die im Kapitel ``Automatisierung der Softwareerstellung'' beschrieben werden. Modularisierung, Konfigurationsbeschreibung, automatische Erstellung und Wiederverwendbarkeit von Artefakten, sowie eine generelle Strategie, die Software so zu entwickeln, dass sie gut testbar ist. Die vollständige Erfüllung dieser Kriterien fordert und fördert eine gute Entwicklungskultur. Dies ist auch der Grund warum Continuous Integration seit den Beschreibungen von Kent Beck so sehr an Akzeptanz und Bedeutung gewonnen hat.

Während die Vorteile von Continuous Integration sehr deutlich und gewichtig sind, gibt es auch Kosten die mit dieser Technik einhergehen. Die komplette Automatisierung erfordert einen durchaus wahrnehmbaren Aufwand in der initialen Einrichtung und einen häufig unterschätzten Aufwand in der Aufrechterhaltung. Gerade bei weniger wichtigen Projekten oder bei Projekten unter besonders hohem Druck, kann es passieren, dass wichtige Bestandteile von Continuous Integration vernachlässigt werden. Dass heißt die Testabdeckung sinkt deutlich oder etwa Teile des Konfigurationsmanagements werden umgangen und essentielle Abhängigkeiten von Hand gepflegt. Dieses Verhalten widerspricht offensichtlich der langfristigen Wertschöpfung und bricht mit den Prinzipien von Continuous Integration. Leider sind aber menschliche Schwächen auch in Continous Integration noch ein Problem, welches nur durch Training und Motivation zur Selbstdisziplin der Entwickler behoben werden kann. In Kapitel~\ref{ch:visu_meth} wird auch das Thema Entwicklerdisziplin erneut aufgegriffen und auf die Gratwanderung von gesunder Entwicklungskultur und destruktiver ``blaming culture'' eingegangen.

\section{Dezentrale Versionierung mit Git}

Im Kapitel~\ref{subsec:konfigurationsverwaltung} Konfigurationsverwaltung wurde bereits auf das Thema Versionsverwaltung eingegangen. Als das populärste System im Open-Source-Bereiche und eines der großen Systeme im kommerziellen Bereich\footcite{g2crowd2018}, wird nun anhand von Git erläutert, wie die dezentrale Versionsverwaltung grundlegend verwendet wird. Außerdem kommen einige git-spezifische Merkmale hinzu, welche bei der visuellen Aufbereitung wichtig werden.

Git wurde vom Linux-Schöpfer Linus Torwalds im April 2005 initiiert, aus der Not heraus sich vom vorher verwendeten ``BitKeeper''-System zu lösen, da diese nicht mehr mit der eigenen Open-Source-Lizenz in Einklang zubringen war. Git wurde von vornherein als verteiltes, vor Verfälschungen sicheres und effizientes Versionsverwaltungssystem entworfen. Diese Eigenschaften und die Software-Plattform ``GitHub'', machten Git in den letzten 13 Jahren zum de-facto Standard für Versionsverwaltung\footcite{heise-torvald-git2015}.

Als dezentrales Versionsverwaltungssystem ist einer der großen Vorteile von Git, dass die vollständige Historie der Dateien lokal verfügbar ist. Während zentrale Versionsverwaltungssysteme immer einen dedizierten Server benötigen, können dezentrale Varianten über Peer-To-Peer-Schnittstellen kommunizieren. Diese Eigenschaft sorgt neben einer hohen Ausfallsicherheit und Informationsredundanz, auch für eine hohe Flexibilität und die Möglichkeit zahlreiche Arbeitsweisen von Softwareentwicklern zu unterstützen.

\subsection{Verwendung}


\subsection{interne Arbeitsweise}

\section{Feature Branches}

Feature Branches ist ein Kunstwortkombination aus der Vermischung von Feature, dem englischen Wort für Merkmal und Branch dem englisch Wort für Zweig, abgeleitet von der Verzweigung in einem Versionsverwaltungssystem. Die Idee hinter einem Feature Branch ist, dass für jedes neue Merkmal, beziehungsweise für jede neue Anforderung ein neuer Branch in der Versionsverwaltung erstellt wird. Ziel der Feature Branches ist die Isolierung von Änderungen, um die anderen Zweige nicht zu blockieren und Änderungen erst nach Test und Abnahme zusammen zu führen. Dabei sollen mehrere Änderungen parallel entwickelt werden können, ohne dabei zu einem bestimmten Zeitpunkt alle Änderungen in einen Hauptzweig übernehmen zu müssen.

Prinzipiell ist diese Technik unabhängig von der verwendeten Versionsverwaltung, erhielt aber vor allem durch dezentrale Versionsverwaltungssysteme an Bedeutung. Der Grund dafür liegt in der deutlich einfacheren und weniger aufwändigen Erstellung von Branches und deren Zusammenführung.

Feature Branching wird kontrovers diskutiert[quote martinfowler feature branch, quote Yegor Bugayenko, quote jamesmckay]. Dabei wird angemahnt, dass die Abspaltung in einen Zweig, das Problem der Zusammenführung von Codeänderungen nicht behebt, sondern verschlimmert. So wird der Aufwand, der benötigt wird um zwei Verzweigungen zusammen zuführen, potentiell immer höher, um so mehr Änderungen hinzukommen. Die Aufwandserhöhung kann soweit gehen, dass ein psychologischer Faktor hinzukommt, der die Entscheidung zur Zusammenführung weiter belastet. Diese Zusammenführung wird dann auch als ``big scary merge'' oder ``big bang merge'' bezeichnet.

Kontrovers dazu wird die Blockade vermeidende Natur von Feature Branches hervorgehoben und die Angst vor dem ``big scary merge'' durch Entwicklerdisziplin und Automatisierung gemildert. Entwicklerdisziplin meint hier Feature Branches nur für kleine Änderungen zu verwenden und durch Hilfe von Automatisierung schnelle Rückmeldung über die Codequalität zu erhalten. Des Weiteren werden durch moderne ``pull based''[quote github about pull requests] Verzweigungsstratgien Entwickler dazu angehalten Codeprüfungen durch andere Entwickler einzufordern. Diese Lösungsstrategien werden in Kapitel~\ref{ch:visu_meth} weiterführend betrachtet.

\section{Visualisierung von Graphen}