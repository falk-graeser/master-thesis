\chapter{Einleitung}

Continuous-Integration ist, spätestens seit der Veröffentlichung \glqq Extreme Programming Explained\grqq{}  von Kent Beck im Jahre 1999,
eine anerkannte Basis für hochqualitative Softwareentwicklung. Knapp 20 Jahre nach dem Werk von Kent Beck ist Continuous-Integration weit
verbreitet und wird vollautomatisiert für Continuous-Deployment genutzt.

Kern von Continuous-Integration ist das kontinuierliche Zusammenführen aller Änderungen, die von Softwareentwicklern eingebracht werden. Continuous-Integration fordert und fördert die Integration dieser Änderungen.
Allerdings fordert gerade diese andauernde und nicht zu umgehende Integration, ein hohes Maß an Kommunikation und erfahrene Softwareentwickler.

Die Anforderung \glqq wann immer der automatische Software-Build fehlschlägt, muss das gesamte Team mit allem stoppen und das Problem sofort beheben\grqq{}\footcite[vgl.][]{humble2010} ist einer der Kernpunkte von Continuous-Integration. Wird das Problem nicht behoben, ist das ganze Team blockiert. Im ungünstigeren Fall werden weiterhin Änderungen zu einem bereits fehlschlagendem Problem hinzugefügt und die Behebung wird zunehmend komplizierter.

Mit dem Erstarken von dezentralen Versionsverwaltungssystemen gibt es eine Alternative zur stetigen Integration. Für jede Anforderung konnte nun mit geringem Aufwand ein eigener Arbeitszweig erstellt werden. Dieser Arbeitszweig wird einzeln entwickelt, getestet und nur im Falle eines positiven Testergebnisses, wieder in den Hauptzweig integriert. Üblicherweise wird für jedes Feature ein solcher Branch erstellt.

Während viele Softwareentwickler dies als kompetente Lösungsstrategie erachten\footcite{ci-is-dead}, sprach sich Martin Fowler im Jahr 2009 intensiv gegen das Feature-Branch-Modell aus\footcite{fowler-feature-branch}.

Während viele moderne und professionelle Lösungen Verfahren für Feature-Branches adaptiert haben und zahlreiche Werkzeuge zur Unterstützung anbieten, gibt es immer noch deutliche Schwierigkeiten. Häufig entwickeln sich \glqq kurzlebige und kleine\grqq{}
Entwicklungszweige zu \glqq großen und langlebigen\grqq{} Veränderungen. Diese betreffen schnell verschiedene Komponenten der Softwareanwendung und werden immer schwieriger zu integrieren.

\section{Zielstellung}

Ziel der Arbeit ist es den Konflikt von \glqq Feature-Branches\grqq{} und \glqq Continuous-Integration\grqq{} aufzuarbeiten, sowie visuelle und methodische Lösungsansätze anzubieten. 

Dazu werden zunächst die Grundlagen und die beiden Begrifflichkeiten selbst erläutert. Es wird auf die automatische Erstellung von lauffähiger Software aus Softwarequellen eingegangen. Die Validierung durch Tests und die Einschätzung der Software durch Software-Metriken wird betrachtet. Schließlich werden Methodiken und Visualisierungen erläutert, welche die beiden vermeintlich konträren Entwicklungsmethoden näher zusammenbringt.

Die Masterarbeit soll sich mit folgenden Thesen auseinander setzen:
\begin{itemize}
\item \glqq Feature-Branches und Continuous-Integration sind unvereinbare Prinzipien.\grqq{},
\item \glqq Continuous-Delivery ist zu komplex, um es in jedem Projekt zu verwenden.\grqq{} und
\item \glqq Der Einsatz von Continuous-Delivery steigert die Qualität des Entwicklungsprozesses.\grqq{}
\end{itemize}
Dabei sollen Indizien oder sogar Beweise jeweils für deren Bestätigung bzw. Falsifizierung geliefert werden.

Weiter werden die Ergebnisse eine Umfrage zu \glqq Nutzungsverhalten von Versionsverwaltungssystemen und Tests\grqq{} erläutert. Die Ergebnisse sollen das Nutzungsverhalten von Versionsverwaltungssystemen und Tests aufzeigen. Als Basis für sowohl Continuous-Integration, als Feature-Branches, können die Ergebnisse der Umfrage eine weitere Perspektive zur Problemstellung beitragen. Schließlich werden anhand eines Prototyp Ansätze für die entwickelten und gesammelten Methoden aufzeigt.

\section{Abgrenzung}

Continuous-Integration und Feature-Branches haben einen starken Einfluss auf den Software-Entwicklungsprozess in dem sie angewendet werden. Daher existieren unter anderem Schnittmengen zu den Themenbereichen Anforderungsmanagement, Psychologie und \\Continuous-Deployment. 
Da der Fokus der Arbeit auf dem Konflikt der beiden Methodiken liegt, werden die benannten Bereiche nur angeschnitten. Das Anforderungsmanagement und die psychologische Komponente sind beide notwendig, für die Durchführung von Continuous-Integration und Feature-Branches. Allerdings üben weder das Anforderungsmanagement, noch psychologische Einflussfaktoren eine deutlichen Einfluss auf den Konflikt zwischen Continuous-Integration und Feature-Branches aus. 

Continuous-Deployment würde ebenso den Rahmen der Betrachtung überschreiten. Continuous-Deployment-Prinzipien greifen primär, nachdem Änderungen von Entwicklern integriert wurden. Als Teilaspekt von Continuous-Deployment, wird Continuous-Delivery stärker beleuchtet.